% !TEX TS-program = xelatex
% !TEX encoding = UTF-8 Unicode

\documentclass[a4paper,preload=gradientframe]{xoblivoir}

\usepackage{fapapersize}
\usefapapersize{*,*,30mm,*,20mm,30mm}

\usepackage{tabu}

\usepackage{longtable}

\ifxetex
\setkormainfont{HCR Batang LVT}
\setkorsansfont{HCR Dotum LVT}
\fi

\begin{document}

\section{숙제 1: 일반적인 사용법, 열 합치기, 선 긋기}

\gradientframe[padding=0mm,linewidth=.4pt]{%
\begin{tabu}{*{4}{X}X[1.5]X[1.5]X[2]}
%\toprule
\multicolumn{7}{c}{2월 중 커피를 사준다는 사람 명단} \\ \midrule
\tabuphantomline
닉네임 & 똘돌 & like샘 & 두텁  & 작은나무 & 록희 & Hoze \\ \tabucline[0.5pt on 2pt off 1.8pt]{-}
이유 & \multicolumn{3}{c}{TnXTeX을 만들어서} & 그냥 사준다고 함 & 옛날에 좋은 일이 있어서 & 여대생 따님*과 초등생 아드님을 두어서 \\
\bottomrule
 & & & & \multicolumn{3}{c}{1월 중 커피를 사준 사람 명단} \\ \tabucline{5-}
\tabuphantomline
 & & & & 닉네임 & ChoF & 그로몹 \\ \tabucline[0.5pt on 2pt off 1.8pt]{5-}
\tabuphantomline
 & & & & 이유 &일과시간에 반차를 쓰고 강의를 들으러 왔다고 & 하필 가장 연장자인 자리에 참석하여 \\ %\tabucline[1pt]{5-}
%\tabuphantomline
\end{tabu}
}

\section{숙제 2: 표 안의 표}

\gradientframe{%
\begin{tabu} {X[c]|X[c]|X[3]|X[3]}
%\bottomrule
\tabuphantomline
\rowfont[c]{} 학습과정 & 학습요소 & \multicolumn{2}{c}{교수·학습 활동} \\ \tabucline{-}
\tabuphantomline
기초 및 \newline 문제 파악 & 지난시간 학습내용 파악 & \textbf{T} 사랑하는 초등학생 여러분. 지난 시간에 무엇을 공부하였나요?  & 
\textbf{일동} (웅성웅성) \newline
\textbf{S1} 직사각형의 둘레의 길이를 구하는 방법에 대해 배웠습니다. \newline
\textbf{S2} 정사각형의 둘레의 길이 구하는 법도 배웠어요.
\\ \tabucline{2-}
& 이번시간 학습내용 파악 & \textbf{T} 아래와 같은 직사각형의 둘레의 길이는 어떻게 구할 수 있습니까? 

\begin{center}
\begin{tabu}spread 2pt{X*{5}{|X}|}
\multicolumn{1}{l}{}& \multicolumn{5}{l}{1}\\\tabucline{2-}
\tabuphantomline
\everyrow{\tabucline{2-}}
1& & & & &\\
& & & & & \\
& & & & & \\
\end{tabu}
\end{center}

\textbf{요점} 직사각형을 단위 정사각형의 모임으로 쪼개고 총합을 구하도록 유도한다. \newline
\indent$5+3+x+y  =\ell$ 
\newline$\Rightarrow (5+3)\times \varepsilon =\ell$(m)
 & \textbf{S1} 가로의 길이가 5이고 세로의 길이가 3이므로 직사각형의 둘레의 길이는 $5+3+5+3=16$입니다. \newline
\textbf{S2} 가로의 길이 5, 세로의 길이 3이고 이것이 두 번 있으므로 직사각형의 둘레의 길이는 $(5+3)\times 2=16$입니다. 
\newline
\textbf{S3} 각본상 $16$입니다.  \\
%\bottomrule
\end{tabu}
}

\end{document}
%%%%%%%%%%%%%%%%%%%%%%%%%%%%%%%%%%%%%


\section{번외 숙제: longtabu 환경}

\begin{longtabu}spread 10pt {X[c]X[2]X[2]X[2]}
\toprule
\rowfont{\sffamily}$x$(도) & $\sin x$ &  $\cos x$ &  $\tan x$  \\
\midrule
\endfirsthead
\toprule
\rowfont{\sffamily}$x$(도) & $\sin x$ &  $\cos x$ &  $\tan x$  \\
\midrule
\endhead
\bottomrule
\endfoot
\bottomrule
\endlastfoot
0 & 0 & 1 & 0 \\ 
1 & 0.0174524 & 0.99984769 & 0.01745506 \\ 
2 & 0.03489949 & 0.99939082 & 0.03492076 \\ 
3 & 0.05233595 & 0.99862953 & 0.05240777 \\ 
4 & 0.06975647 & 0.99756405 & 0.06992681 \\ 
5 & 0.08715574 & 0.99619469 & 0.08748866 \\ 
6 & 0.10452846 & 0.99452189 & 0.10510423 \\ 
7 & 0.12186934 & 0.99254615 & 0.12278456 \\ 
8 & 0.1391731 & 0.99026806 & 0.14054083 \\ 
9 & 0.15643446 & 0.98768834 & 0.15838444 \\ 
10 & 0.17364817 & 0.98480775 & 0.17632698 \\ 
11 & 0.19080899 & 0.98162718 & 0.1943803 \\ 
12 & 0.20791169 & 0.9781476 & 0.21255656 \\ 
13 & 0.22495105 & 0.97437006 & 0.23086819 \\ 
14 & 0.24192189 & 0.97029572 & 0.249328 \\ 
15 & 0.25881904 & 0.96592582 & 0.26794919 \\ 
16 & 0.27563735 & 0.96126169 & 0.28674538 \\ 
17 & 0.2923717 & 0.95630475 & 0.30573068 \\ 
18 & 0.30901699 & 0.95105651 & 0.32491969 \\ 
19 & 0.32556815 & 0.94551857 & 0.34432761 \\ 
20 & 0.34202014 & 0.93969262 & 0.36397023 \\ 
21 & 0.35836794 & 0.93358042 & 0.38386403 \\ 
22 & 0.37460659 & 0.92718385 & 0.40402622 \\ 
23 & 0.39073112 & 0.92050485 & 0.42447481 \\ 
24 & 0.40673664 & 0.91354545 & 0.44522868 \\ 
25 & 0.42261826 & 0.90630778 & 0.46630765 \\ 
26 & 0.43837114 & 0.89879404 & 0.48773258 \\ 
27 & 0.45399049 & 0.89100652 & 0.50952544 \\ 
28 & 0.46947156 & 0.88294759 & 0.53170943 \\ 
29 & 0.48480962 & 0.8746197 & 0.55430905 \\ 
30 & 0.49999999 & 0.8660254 & 0.57735026 \\ 
31 & 0.51503807 & 0.8571673 & 0.60086061 \\ 
32 & 0.52991926 & 0.84804809 & 0.62486935 \\ 
33 & 0.54463903 & 0.83867056 & 0.64940759 \\ 
34 & 0.5591929 & 0.82903757 & 0.67450851 \\ 
35 & 0.57357643 & 0.81915204 & 0.70020753 \\ 
36 & 0.58778525 & 0.80901699 & 0.72654252 \\ 
37 & 0.60181502 & 0.79863551 & 0.75355405 \\ 
38 & 0.61566147 & 0.78801075 & 0.78128562 \\ 
39 & 0.62932039 & 0.77714596 & 0.80978403 \\ 
40 & 0.6427876 & 0.76604444 & 0.83909963 \\ 
41 & 0.65605902 & 0.75470958 & 0.86928673 \\ 
42 & 0.6691306 & 0.74314482 & 0.90040404 \\ 
43 & 0.68199836 & 0.7313537 & 0.93251508 \\ 
44 & 0.69465837 & 0.7193398 & 0.96568877 \\ 
45 & 0.70710678 & 0.70710678 & 0.99999999 \\ 
46 & 0.7193398 & 0.69465837 & 1.03553031 \\ 
47 & 0.7313537 & 0.68199836 & 1.0723687 \\ 
48 & 0.74314482 & 0.6691306 & 1.11061251 \\ 
49 & 0.75470958 & 0.65605902 & 1.1503684 \\ 
50 & 0.76604444 & 0.6427876 & 1.19175359 \\ 
51 & 0.77714596 & 0.62932039 & 1.23489715 \\ 
52 & 0.78801075 & 0.61566147 & 1.27994163 \\ 
53 & 0.79863551 & 0.60181502 & 1.32704482 \\ 
54 & 0.80901699 & 0.58778525 & 1.37638192 \\ 
55 & 0.81915204 & 0.57357643 & 1.428148 \\ 
56 & 0.82903757 & 0.5591929 & 1.48256096 \\ 
57 & 0.83867056 & 0.54463903 & 1.53986496 \\ 
58 & 0.84804809 & 0.52991926 & 1.60033452 \\ 
59 & 0.8571673 & 0.51503807 & 1.66427948 \\ 
60 & 0.8660254 & 0.5 & 1.7320508 \\ 
61 & 0.8746197 & 0.48480962 & 1.80404775 \\ 
62 & 0.88294759 & 0.46947156 & 1.88072646 \\ 
63 & 0.89100652 & 0.45399049 & 1.9626105 \\ 
64 & 0.89879404 & 0.43837114 & 2.05030384 \\ 
65 & 0.90630778 & 0.42261826 & 2.14450692 \\ 
66 & 0.91354545 & 0.40673664 & 2.24603677 \\ 
67 & 0.92050485 & 0.39073112 & 2.35585236 \\ 
68 & 0.92718385 & 0.37460659 & 2.47508685 \\ 
69 & 0.93358042 & 0.35836794 & 2.60508906 \\ 
70 & 0.93969262 & 0.34202014 & 2.74747741 \\ 
71 & 0.94551857 & 0.32556815 & 2.90421087 \\ 
72 & 0.95105651 & 0.30901699 & 3.07768353 \\ 
73 & 0.95630475 & 0.2923717 & 3.27085261 \\ 
74 & 0.96126169 & 0.27563735 & 3.48741444 \\ 
75 & 0.96592582 & 0.25881904 & 3.7320508 \\ 
76 & 0.97029572 & 0.24192189 & 4.01078093 \\ 
77 & 0.97437006 & 0.22495105 & 4.33147587 \\ 
78 & 0.9781476 & 0.20791169 & 4.7046301 \\ 
79 & 0.98162718 & 0.19080899 & 5.14455401 \\ 
80 & 0.98480775 & 0.17364817 & 5.67128181 \\ 
81 & 0.98768834 & 0.15643446 & 6.31375151 \\ 
82 & 0.99026806 & 0.1391731 & 7.11536972 \\ 
83 & 0.99254615 & 0.12186934 & 8.14434642 \\ 
84 & 0.99452189 & 0.10452846 & 9.51436445 \\ 
85 & 0.99619469 & 0.08715574 & 11.43005229 \\ 
86 & 0.99756405 & 0.06975647 & 14.30066624 \\ 
87 & 0.99862953 & 0.05233595 & 19.08113667 \\ 
88 & 0.99939082 & 0.03489949 & 28.63625324 \\ 
89 & 0.99984769 & 0.0174524 & 57.28996148 \\ 
90 & 1 & 0 & $\infty$\\
\end{longtabu}

\end{document}
